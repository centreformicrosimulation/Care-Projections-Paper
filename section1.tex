\section{Introduction}
\label{sec1}

Giving and receiving care are defining features of life. They shape who we are and can profoundly affect diverse aspects that bear on life quality, including the two key margins of economic decision making: labour-leisure and consumption-saving. Care dynamics have garnered increasing attention, as the demographic transition towards older populations, which has been taking place throughout OECD countries, is structurally altering the demand and supply of care. \par
As of 2024, the OECD average share of the population aged 65 or above was 18.51\% (\textcite{OECD:2024}). In the UK, this proportion reaches almost 19\%, showing a dramatic century-long increasing trend. At the turn of the 20th century, approximately 5\% of the UK population was aged 65 or over, and less than 2 in every thousand were aged 85 or over. Since then, the proportion of the population aged 65 or over has increased by almost four times by 2024, and by more than ten times (2.5\%) for the population aged 85 or over. Furthermore, the trend towards an older population is projected to continue into the next century, with the proportion of people aged 65 and over projected to exceed 30\% in the early 2100s, and those aged 85 and over projected to account for almost 1 in 10 people. \par
In the present study, we aim to project the effects of alterations in the old-age dependency ratio\footnote{The old-age dependency ratio is the share of people aged 65 and over (dependents) on the working-age population (typically 15-64). It measures how many older individuals rely on each worker.} on both the receipt and provision of care in the UK. We investigate the impact on the level and cost of care, as well as its share of total GDP, through the life course and across income and wealth distributions. Changes in the old-age dependency ratio are simulated through changes in fertility or mortality rates. Given the primary role of partners in informal care, we also design scenarios where the probability of living in a couple is affected.
SimPaths, an open-source dynamic microsimulation model, is employed to simulate these different counterfactuals over a half-century period. This framework uses data from the UK Household Longitudinal Study (UKHLS) and the Family Resources Survey (FRS) to project life histories over time, developing detailed representations of career paths, family and intergenerational relationships, health, and financial circumstances. \noteDS{Our estimates show that the value of care, as a share of GDP, almost doubles over the five decades of our analysis, with informal care accounting for most of the projected rise.}[Update] \par
% The trend toward older populations has been driven primarily by increasing longevity, reinforced by declining fertility. Life expectancy at birth in the UK increased from 66.2 (70.9) years for men (women) born in 1950 to 79.0 (82.7) years in 2012.  Similar increases in life expectancy have been observed in North America, and slightly higher increases in other Western European countries. Furthermore, although increases in life expectancy in the UK plateaued during the decade to 2023, official projections anticipate a return to growth of approximately 0.1 year of life for each consecutive cohort born between 2024 and 2070.  
% At the same time, fertility rates in the UK fell precipitously from just under 2.93 children per woman in 1964 to 1.69 in 1977 and have since displayed a moderate downward trend to 1.57 in 2022.  The rates since 1977 have been well below replacement (2.1), which exaggerates population ageing. Furthermore, the persistence of recent shifts in fertility has come as some surprise to social planners. The Office for National Statistics (ONS) had projected that the UK total fertility rate in 2040 would have been 1.89 when projections were made in 2014, falling to 1.84 in the 2016 projections, 1.78 in the 2018 projections, and 1.57 in 2020 projections.  
% The shifts in longevity, fertility, and population ageing outlined above have wide-ranging implications for social planning.
The largest strand of the related literature has so far focussed on the implications for care arrangements. \textcite{OECD:2022}, for example, projects that public spending on long-term care across the 27 EU countries will increase from 1.7\% of GDP in 2019 to 2.8\% in 2070. Projected increases to 2070 vary widely by country, from near zero in Greece, Latvia and Bulgaria, up to 2.7\% in the Netherlands and 3.4\% in Denmark. In the UK, the Office for Budget Responsibility (\textcite{OBR:2024}) reports that adult social care spending is projected to rise from 1.5\% of GDP in 2028/29 to 2.4\% by 2073/74. This increase is attributed to ``a combination of demographic pressures and real-terms unit cost growth''. According to \textcite{IFS:2018}, to keep up with current demographic trends, social care funding should increase by 3.9\% a year across the UK over the next 15 years. Similarly, \textcite{rocks:2021} estimate an average annual increase in funding for social care between 4.3\% and 5.8\% during the period 2019--2031, depending on the considered scenario. \par
The modelling work underlying the projections outlined above is often thinly documented. In the case of the UK, for example, much of the underlying modelling work has been conducted using models developed at the Personal Social Services Research Unit (PSSRU), whose most recent publicly available description is in \textcite{wittenberg:2006}.\footnote{See \textcite{EC:2021} for a description of the models used in the \textcite{OECD:2022} projections.} However, the existing literature suggests that these models share a common analytical approach. That is, they combine existing population projections with statistical descriptions concerning the incidence of care and exogenous assumptions about how care needs will evolve into the future. Key assumptions underlying the \textcite{OECD:2022} projections, for example, are that ``half of the future gains in life expectancy are spent in good health and an income elasticity of health care spending is converging linearly from 1.1 in 2019 to unity in 2070''. Such modelling assumptions help to provide a ``statistical projection'' of what care needs may be into the future and have the advantage of connecting in an ostensibly transparent fashion disparate statistical evidence to obtain inferences for tertiary subjects of interest. Yet, while the abstractions associated with such methods are generally transparent, they also risk obscuring important features concerning the influence of caring through the life course. In fact, although care has its most obvious consequences when it is actually required, its effects are likely to extend to other periods of life. People may anticipate the need to provide informal care in response to deteriorating health of loved ones. Similarly, a reason given for high savings rates among the elderly is the desire to self-insure against the needs consequent on adverse health shocks, including the need for (expensive) formal care.  Furthermore, both informal care and incapacity of demanding care can have effects that persist well after the actual episodes of care have ended due, for example, to labour market scarring or depleted savings. \par
This is the first paper that explores the life-course effects associated with the demand and supply of care. The life-course perspective considered for this study is designed to account for \textit{ex-ante} effects associated with anticipation of the possibility of future care needs and responsibilities, the influence on individual circumstances while care is needed, as well as \textit{ex-post} effects after care needs have passed. The effects of population aging on care are explored by comparing the baseline projections against projections generated under alternative sets of assumptions on the evolution of fertility and mortality rates. Comparisons of interest include the effects on demand and supply of care through the life-course, the overall cost of social care, and its share of total GDP. \par
% \textcolor{red}{All materials used for the study are fully open-source, and a step-by-step walk-through to replicate reported results is provided in Appendix E}. The remainder of the study is organised as follows. Section \ref{sec2} discusses the statistical background for the study, focussing on the influence of alternative forms of care on the life course. The approach taken to model care is described in Section \ref{sec3} and results are reported in Section \ref{sec4}. Section \ref{sec5} concludes.